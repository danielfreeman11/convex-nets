\section{Proofs}

\subsection{Proof of Proposition \ref{localdistprop}}

Let 
$$A(w_1, w_2) = \{ x \in \R^n; \, \langle x, w_1 \rangle \geq 0\,,\, \langle x, w_2 \rangle \geq 0\}~.$$
By definition, we have 
\begin{eqnarray}
\label{col1}
\langle w_1, w_2 \rangle_Z &=& \E \{ \max(0, \langle X, w_1 \rangle ) \max(0, \langle X, w_2 \rangle ) \} \\
&=& \int_{A(w_1, w_2)} \langle x, w_1 \rangle  \langle x, w_2 \rangle dP(x)~, \\
&=& \int_{Q({A}(w_1, w_2))}  \langle Q(x), w_1 \rangle  \langle Q(x), w_2 \rangle (d\bar{P}(Q(x)))~,  
\end{eqnarray}
where $Q$ is the orthogonal projection onto the space spanned by $w_1$ and $w_2$ and
 $d\bar{P}(x)=d\bar{P}(x_1, x_2)$ is the marginal density on that subspace. 
 Since this projection does not interfere with the rest of the proof, we abuse notation by dropping the $Q$ and still referring to $dP(x)$ as the probability density.

Now, let $r = \frac{1}{2}\| w_1 + w_2 \| = \frac{1 + \cos(\alpha)}{2}$ and $d = \frac{w_2 - w_1}{2}$.
By construction we have 
$$w_1 = r w_m - d~,~ w_2 = r w_m + d~,$$
and thus 
\begin{equation}
\label{col2}
\langle x, w_1 \rangle  \langle x, w_2 \rangle = r^2 | \langle x, w_m \rangle |^2 - | \langle x, d \rangle |^2~.
\end{equation}
By denoting $C(w_m) = \{ x \in \R^n;\, \langle x, w_m \rangle \geq 0 \}$, 
observe that $A(w_1, w_2 ) \subseteq C(w_m)$. Let us denote by $B = C(w_m) \setminus A(w_1, w_2) $ the disjoint complement. It results that 
\begin{eqnarray}
\label{col5}
\langle w_1, w_2 \rangle_Z &=& \int_{A(w_1, w_2)} \langle x, w_1 \rangle  \langle x, w_2 \rangle dP(x) \nonumber \\
&=& \int_{C(w_m)} [r^2 | \langle x, w_m \rangle |^2 - | \langle x, d \rangle |^2 ] dP(x) - r^2 \int_B  | \langle x, w_m \rangle |^2 dP(x) + \int_B  | \langle x, d \rangle |^2  dP(x) \nonumber \\ 
&=& r^2 \| w_m \|_Z^2 - \underbrace{ r^2 \int_B  | \langle x, w_m \rangle |^2 dP(x)}_{E_1} - \underbrace{\int_{A(w_1, w_2)} | \langle x, d \rangle |^2  dP(x) }_{E_2}~. 
\end{eqnarray} 
We conclude by bounding each error term $E_1$ and $E_2$ separately:
\begin{equation}
\label{col3}
0 \leq E_1 \leq r^2 |\sin(\alpha)|^2 \int_B \| x \|^2 dP(x) \leq r^2 |\sin(\alpha)|^2 2 \| \Sigma_X\|~,
\end{equation}
since every point in $B$ by definition has angle greater than $\pi/2 - \alpha$ from $w_m$. Also,
\begin{equation}
\label{col4}
0 \leq E_2 \leq \|d \|^2 \int_{A(w_1, w_2)} \| x \|^2 dP(x) \leq \frac{1 - \cos(\alpha)}{2} 2 \| \Sigma_X \|
\end{equation}
by direct application of Cauchy-Schwartz. The proof is completed by plugging the bounds from (\ref{col3}) and (\ref{col4}) into (\ref{col5})  $\square$.


\subsection{Proof of Theorem \ref{maintheo}}

A path from $\theta^A$ to $\theta^B$ will be constructed 
as follows:
\begin{enumerate}
\item from $\theta^A$ to $\theta_{lA}$, the 
best linear predictor using the same first layer. 
\item from $\theta_{lA}$ to $\theta_{sA}$, the best $(M-n)$-term approximation using perturbed 
atoms, 
\item from $\theta_{sA}$ to $\theta^*$ the oracle $n$ term approximation,  
\item from $\theta^*$ to $\theta_{sB}$,
\item from $\theta_{sB}$ to $\theta_{lB}$,
\item from $\theta_{lB}$ to $\theta^{B}$.
\end{enumerate}
The subpaths (1) and (6) only involve changing the parameters of the second layer, 
which define a convex loss. Therefore a linear path is sufficient.
Subpaths (3) and (4) can also be constructed using only parameters of the second layer, 
by observing that one can fit into a single $n \times M$ parameter matrix both the 
$(M-n)$-term approximation and the best $n$-term approximation. A linear path is therefore 
also sufficient. 

We finally need to show how to construct the subpaths (2) and (5).
Let $\tilde{W}_A$ be the resulting perturbed first-layer parameter matrix 
with $M-n$ sparse coefficients $\gamma_A$.
Let us consider an auxiliary regression of the form 
$$\overline{W} = [ W^A ; \tilde{W}_A] ~\in \R^{n \times 2M}~.$$
and regression parameters 
$$\overline{\beta}_1 = [ \beta_1; 0]~,~\overline{\beta}_2 = [0; \gamma_A]~.$$
Clearly 
$$\E\{ | Y - \overline{\beta}_1 \overline{W} |^2 \} + \kappa \| \overline{\beta}_1 \|^2 = \E\{ | Y - \beta_1 W^A |^2 \} + \kappa \| {\beta}_1 \|^2 $$ 
and similarly for $\overline{\beta}_2$. The augmented linear path $\eta(t) =(1- t) \overline{\beta}_1 + t \overline{\beta}_2$ thus satisfies 
$$\forall~t~,\overline{L}(t) = \E\{ | Y - \eta(t) \overline{W} |^2 \} + \kappa \| \eta(t) \|^2 \leq \max(\overline{L}(0), \overline{L}(1))~. $$
Let us now approximate this augmented linear path with a path in terms of first and second layer weights. 
We consider
$$\eta_1(t) = (1-t) W^A + t \tilde{W}_A~,\text{ and}~\eta_2(t) = (1- t) {\beta}_1 + t \gamma_A~.$$
We have that 
\begin{eqnarray}
\label{bub1}
\Forr(\{ \eta_1(t), \eta_2(t) \}) &=& \E \{ | Y - \eta_2(t) Z(\eta_1(t) ) |^2 \} + \kappa \| \eta_2(t) \|^2  \\ 
&\leq & \E \{ | Y - \eta_2(t) Z(\eta_1(t) ) |^2 \} + \kappa(  ( 1-t) \| {\beta}_1\|^2 + t \| \gamma_A \|^2 ) \nonumber \\
& = & \overline{L}(t) + \E \{ | Y - \eta_2(t) Z(\eta_1(t) ) |^2 \}  - \E \{ | Y - (1-t) \beta_1 Z(W^A) - t \gamma_A Z(\tilde{W}_A) |^2 \} ~.  \nonumber
\end{eqnarray}
Finally, we verify that
{\small 
\begin{eqnarray}
\label{bub2}
& \left | \E \{ | Y - \eta_2(t) Z(\eta_1(t) ) |^2 \}  - \E \{ | Y - (1-t) \beta_1 Z(W^A) - t \gamma_A Z(\tilde{W}_A) |^2 \} \right|  \leq\\
& \leq 4  \alpha \max(\E |Y|^2, \sqrt{\E|Y^2|}) \| \Sigma_X \| ( \kappa^{-1/2} + M \alpha \sqrt{\E|Y^2|} \kappa^{-1}) + O(\alpha^2)~. \nonumber
\end{eqnarray}}
Indeed, from Proposition \ref{localdistprop}, and using the fact that 
$$\forall~i\leq M,\, t \in [0,1]~,~\left| \angle( (1-t)w^A_i + t \tilde{w}^A_i ; w^A_i) \right| \leq \alpha~,~ \left| \angle( (1-t)w^A_i + t \tilde{w}^A_i ; \tilde{w}^A_i) \right| \leq \alpha $$
we can write 
$$(1-t) \beta_{1,i} z(w^A_i) - t \gamma_{A,i} z(\tilde{w}^A_i) \stackrel{d}{=} \eta_2(t)_i z(\eta_1(t)_i) + n_i ~,$$
with $\E\{ |n_i |^2 \} \leq 4 |\eta_2(t)_i|^2 \| \Sigma_X \| \alpha^2 + O(\alpha^4)~$ and $\E |n_i| \leq 2 |\eta_2(t)_i| \alpha \sqrt{\| \Sigma_X\|}$ using concavity of the moments.
Thus 
\begin{eqnarray*}
&& \left | \E \{ | Y - \eta_2(t) Z(\eta_1(t) ) |^2 \}  - \E \{ | Y - (1-t) \beta_1 Z(W^A) - t \gamma_A Z(\tilde{W}_A) |^2 \} \right| \\
 &\leq& 2\E \left\{  \sum_i (Y - \eta_2(t) Z(\eta_1(t) )) n_i  \right\} + \E \left\{ | \sum_i n_i |^2 \right\} \\
 &\leq & 4\left(\alpha \sqrt{\E|Y^2|} \| \Sigma_X\|  \| \eta_2 \| + \alpha^2 (\| \eta_2 \|_1)^2  \| \Sigma_X \| \right) \\
 &\leq & 4 \alpha \max(1, \sqrt{\E|Y^2|}) \| \Sigma_X \| ( \| \eta_2 \| + M \alpha \| \eta_2 \|^2) + O(\alpha^2) \\
 & \leq & 4 \alpha \max(\sqrt{\E|Y^2|}, {\E|Y^2|}) \| \Sigma_X \|  ( \kappa^{-1/2} + M \alpha \sqrt{\E|Y^2|} \kappa^{-1}) + O(\alpha^2)~,
\end{eqnarray*}
 which proves (\ref{bub2}). 
 
We have just constructed a path from $\theta^A$ to $\theta^B$, in which all subpaths except (2) and (5) have energy maximized at the extrema due to convexity. For these two subpaths, (\ref{bub2}) shows that it is sufficient to add the corresponding upper bound on the linear subpath. This concludes the proof. $\square$
 
%8 \sqrt{\E|Y^2} \} \max( \| \beta \|^2, \| \gamma_A \|^2) \alpha^2 \| \Sigma_X \| 



% and  $\| \beta \| \leq \kappa^{-1}$  $\square$.

\subsection{Proof of Corollary \ref{maincoro}}


Use pigeonhole principle to control how many directions are within an angle smaller than $\epsilon$. 





